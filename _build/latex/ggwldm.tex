%% Generated by Sphinx.
\def\sphinxdocclass{report}
\documentclass[letterpaper,10pt,english]{sphinxmanual}
\ifdefined\pdfpxdimen
   \let\sphinxpxdimen\pdfpxdimen\else\newdimen\sphinxpxdimen
\fi \sphinxpxdimen=.75bp\relax
\ifdefined\pdfimageresolution
    \pdfimageresolution= \numexpr \dimexpr1in\relax/\sphinxpxdimen\relax
\fi
%% let collapsible pdf bookmarks panel have high depth per default
\PassOptionsToPackage{bookmarksdepth=5}{hyperref}

\PassOptionsToPackage{booktabs}{sphinx}
\PassOptionsToPackage{colorrows}{sphinx}

\PassOptionsToPackage{warn}{textcomp}
\usepackage[utf8]{inputenc}
\ifdefined\DeclareUnicodeCharacter
% support both utf8 and utf8x syntaxes
  \ifdefined\DeclareUnicodeCharacterAsOptional
    \def\sphinxDUC#1{\DeclareUnicodeCharacter{"#1}}
  \else
    \let\sphinxDUC\DeclareUnicodeCharacter
  \fi
  \sphinxDUC{00A0}{\nobreakspace}
  \sphinxDUC{2500}{\sphinxunichar{2500}}
  \sphinxDUC{2502}{\sphinxunichar{2502}}
  \sphinxDUC{2514}{\sphinxunichar{2514}}
  \sphinxDUC{251C}{\sphinxunichar{251C}}
  \sphinxDUC{2572}{\textbackslash}
\fi
\usepackage{cmap}
\usepackage[T1]{fontenc}
\usepackage{amsmath,amssymb,amstext}
\usepackage{babel}



\usepackage{tgtermes}
\usepackage{tgheros}
\renewcommand{\ttdefault}{txtt}



\usepackage[Bjarne]{fncychap}
\usepackage{sphinx}

\fvset{fontsize=auto}
\usepackage{geometry}


% Include hyperref last.
\usepackage{hyperref}
% Fix anchor placement for figures with captions.
\usepackage{hypcap}% it must be loaded after hyperref.
% Set up styles of URL: it should be placed after hyperref.
\urlstyle{same}

\addto\captionsenglish{\renewcommand{\contentsname}{Contents:}}

\usepackage{sphinxmessages}
\setcounter{tocdepth}{1}



\title{GGW LDM}
\date{Oct 10, 2025}
\release{0.0.1}
\author{OSS}
\newcommand{\sphinxlogo}{\vbox{}}
\renewcommand{\releasename}{Release}
\makeindex
\begin{document}

\ifdefined\shorthandoff
  \ifnum\catcode`\=\string=\active\shorthandoff{=}\fi
  \ifnum\catcode`\"=\active\shorthandoff{"}\fi
\fi

\pagestyle{empty}
\sphinxmaketitle
\pagestyle{plain}
\sphinxtableofcontents
\pagestyle{normal}
\phantomsection\label{\detokenize{index::doc}}


\sphinxAtStartPar
Bienvenue dans la documentation de GGW Land Degradation Management

\sphinxstepscope


\chapter{Introduction}
\label{\detokenize{introduction:introduction}}\label{\detokenize{introduction::doc}}
\begin{figure}[H]
\centering
\capstart

\noindent\sphinxincludegraphics[width=600\sphinxpxdimen]{{Home_page}.png}
\caption{Home page}\label{\detokenize{introduction:id1}}\end{figure}

\sphinxAtStartPar
La plateforme régionale de suivi de la dégradation des terres et de la gestion durable des terres (GDT) est développée dans le cadre du projet GEF\sphinxhyphen{}9825, mis en œuvre par le PNUE avec l’appui du FEM et coordonné par l’Observatoire du Sahara et du Sahel (OSS).
Elle vise à appuyer les pays de la Grande Muraille Verte (Burkina Faso, Éthiopie, Niger et Sénégal) dans le suivi, l’analyse et la diffusion des informations sur la dégradation des terres et les pratiques de GDT.
Hébergée par l’OSS, la plateforme permet de centraliser les données, visualiser les tendances, et faciliter la planification et la prise de décision en matière de restauration des terres.


\section{Caractéristiques principales}
\label{\detokenize{introduction:caracteristiques-principales}}\begin{itemize}
\item {} 
\sphinxAtStartPar
Calcul d’indicateurs environnementaux

\item {} 
\sphinxAtStartPar
Accès à diverses sources de données

\item {} 
\sphinxAtStartPar
Outils de cartographie avancés

\item {} 
\sphinxAtStartPar
Interface utilisateur intuitive

\end{itemize}

\noindent{\hspace*{\fill}\sphinxincludegraphics[width=600\sphinxpxdimen]{{geoportail_interface}.png}\hspace*{\fill}}


\section{Visitez la plateforme}
\label{\detokenize{introduction:visitez-la-plateforme}}
\sphinxAtStartPar
Pour plus d’informations, veuillez consulter notre site web:
\sphinxurl{https://ggw-ldn.oss-online.org/}

\sphinxstepscope


\chapter{Geoportal User Guide}
\label{\detokenize{plateforme_use:geoportal-user-guide}}\label{\detokenize{plateforme_use::doc}}
\sphinxAtStartPar
This guide explains the main features of the Geoportal interface and how to interact with the map, layers, and analytical tools.

\begin{sphinxcontents}
\begin{itemize}
\item {} 
\sphinxAtStartPar
\phantomsection\label{\detokenize{plateforme_use:id14}}{\hyperref[\detokenize{plateforme_use:geoportal-overview-interface}]{\sphinxcrossref{Geoportal Overview Interface}}}

\item {} 
\sphinxAtStartPar
\phantomsection\label{\detokenize{plateforme_use:id15}}{\hyperref[\detokenize{plateforme_use:change-the-base-map}]{\sphinxcrossref{Change the Base Map}}}

\item {} 
\sphinxAtStartPar
\phantomsection\label{\detokenize{plateforme_use:id16}}{\hyperref[\detokenize{plateforme_use:change-indicators}]{\sphinxcrossref{Change Indicators}}}

\item {} 
\sphinxAtStartPar
\phantomsection\label{\detokenize{plateforme_use:id17}}{\hyperref[\detokenize{plateforme_use:choose-an-area-of-interest}]{\sphinxcrossref{Choose an Area of Interest}}}

\item {} 
\sphinxAtStartPar
\phantomsection\label{\detokenize{plateforme_use:id18}}{\hyperref[\detokenize{plateforme_use:apply-filters}]{\sphinxcrossref{Apply Filters}}}

\item {} 
\sphinxAtStartPar
\phantomsection\label{\detokenize{plateforme_use:id19}}{\hyperref[\detokenize{plateforme_use:use-the-legend}]{\sphinxcrossref{Use the Legend}}}

\item {} 
\sphinxAtStartPar
\phantomsection\label{\detokenize{plateforme_use:id20}}{\hyperref[\detokenize{plateforme_use:view-statistics}]{\sphinxcrossref{View Statistics}}}

\item {} 
\sphinxAtStartPar
\phantomsection\label{\detokenize{plateforme_use:id21}}{\hyperref[\detokenize{plateforme_use:zoom-controls}]{\sphinxcrossref{Zoom Controls}}}

\item {} 
\sphinxAtStartPar
\phantomsection\label{\detokenize{plateforme_use:id22}}{\hyperref[\detokenize{plateforme_use:summary}]{\sphinxcrossref{Summary}}}

\end{itemize}
\end{sphinxcontents}

\sphinxAtStartPar
—


\section{Geoportal Overview Interface}
\label{\detokenize{plateforme_use:geoportal-overview-interface}}\label{\detokenize{plateforme_use:id1}}
\begin{figure}[H]
\centering
\capstart

\noindent\sphinxincludegraphics[width=600\sphinxpxdimen]{{geoportail_interface}.png}
\caption{Overview of the Geoportal interface}\label{\detokenize{plateforme_use:id6}}\end{figure}

\sphinxAtStartPar
When you open the Geoportal, you will see the main map view, toolbar, and side panels for data and analysis.
This layout allows easy navigation and quick access to both datasets and analytical tools.

\sphinxAtStartPar
{[}  Back to top{]}(\#geoportal\sphinxhyphen{}user\sphinxhyphen{}guide)

\sphinxAtStartPar
—


\section{Change the Base Map}
\label{\detokenize{plateforme_use:change-the-base-map}}\label{\detokenize{plateforme_use:change-base-map}}
\begin{figure}[H]
\centering
\capstart

\noindent\sphinxincludegraphics[width=300\sphinxpxdimen]{{change_base_map}.png}
\caption{Changing the base map}\label{\detokenize{plateforme_use:id7}}\end{figure}

\sphinxAtStartPar
Users can switch between different base maps (e.g., satellite, topographic, street view)
to visualize spatial information according to their preference.
The base map selector is located on the right side of the toolbar.

\sphinxAtStartPar
{[}  Back to top{]}(\#geoportal\sphinxhyphen{}user\sphinxhyphen{}guide)

\sphinxAtStartPar
—


\section{Change Indicators}
\label{\detokenize{plateforme_use:change-indicators}}\label{\detokenize{plateforme_use:id2}}
\begin{figure}[H]
\centering
\capstart

\noindent\sphinxincludegraphics[width=300\sphinxpxdimen]{{change_indicators}.png}
\caption{Selecting an indicator}\label{\detokenize{plateforme_use:id8}}\end{figure}

\sphinxAtStartPar
Click on the indicator panel to display available datasets, such as land degradation indicators or productivity layers.
Indicators can be combined or compared visually on the map.

\sphinxAtStartPar
{[}  Back to top{]}(\#geoportal\sphinxhyphen{}user\sphinxhyphen{}guide)

\sphinxAtStartPar
—


\section{Choose an Area of Interest}
\label{\detokenize{plateforme_use:choose-an-area-of-interest}}\label{\detokenize{plateforme_use:choose-area-of-interest}}
\begin{figure}[H]
\centering
\capstart

\noindent\sphinxincludegraphics[width=300\sphinxpxdimen]{{choose_area_of_interest}.png}
\caption{Defining the area of interest}\label{\detokenize{plateforme_use:id9}}\end{figure}

\sphinxAtStartPar
You can select a specific region or administrative boundary to focus the analysis.
This helps restrict calculations and visualization to your study area.

\sphinxAtStartPar
{[}  Back to top{]}(\#geoportal\sphinxhyphen{}user\sphinxhyphen{}guide)

\sphinxAtStartPar
—


\section{Apply Filters}
\label{\detokenize{plateforme_use:apply-filters}}\label{\detokenize{plateforme_use:id3}}
\begin{figure}[H]
\centering
\capstart

\noindent\sphinxincludegraphics[width=300\sphinxpxdimen]{{choose_filter}.png}
\caption{Applying filters}\label{\detokenize{plateforme_use:id10}}\end{figure}

\sphinxAtStartPar
Use the filter panel to refine your data selection (e.g., time period, indicator type, or data source).

\sphinxAtStartPar
{[}  Back to top{]}(\#geoportal\sphinxhyphen{}user\sphinxhyphen{}guide)

\sphinxAtStartPar
—


\section{Use the Legend}
\label{\detokenize{plateforme_use:use-the-legend}}\label{\detokenize{plateforme_use:use-legend}}
\begin{figure}[H]
\centering
\capstart

\noindent\sphinxincludegraphics[width=100\sphinxpxdimen]{{open_legend}.png}
\caption{Opening and understanding the legend}\label{\detokenize{plateforme_use:id11}}\end{figure}

\sphinxAtStartPar
The legend explains the meaning of colors and symbols on the map, helping you interpret the spatial data correctly.

\sphinxAtStartPar
{[}  Back to top{]}(\#geoportal\sphinxhyphen{}user\sphinxhyphen{}guide)

\sphinxAtStartPar
—


\section{View Statistics}
\label{\detokenize{plateforme_use:view-statistics}}\label{\detokenize{plateforme_use:id4}}
\begin{figure}[H]
\centering
\capstart

\noindent\sphinxincludegraphics[width=100\sphinxpxdimen]{{open_stats}.png}
\caption{Viewing map statistics}\label{\detokenize{plateforme_use:id12}}\end{figure}

\sphinxAtStartPar
The statistics panel displays charts and numerical summaries for the selected area or indicator.

\sphinxAtStartPar
{[}  Back to top{]}(\#geoportal\sphinxhyphen{}user\sphinxhyphen{}guide)

\sphinxAtStartPar
—


\section{Zoom Controls}
\label{\detokenize{plateforme_use:zoom-controls}}\label{\detokenize{plateforme_use:id5}}
\begin{figure}[H]
\centering
\capstart

\noindent\sphinxincludegraphics[width=100\sphinxpxdimen]{{zoom_buttons}.png}
\caption{Zooming in and out}\label{\detokenize{plateforme_use:id13}}\end{figure}

\sphinxAtStartPar
Use the zoom buttons or mouse wheel to explore specific regions of the map more closely.

\sphinxAtStartPar
{[}  Back to top{]}(\#geoportal\sphinxhyphen{}user\sphinxhyphen{}guide)

\sphinxAtStartPar
—


\section{Summary}
\label{\detokenize{plateforme_use:summary}}
\sphinxAtStartPar
This view summarizes all components of the Geoportal, including tools, panels, and map display.

\sphinxstepscope


\chapter{SDG 15.3.1}
\label{\detokenize{SDGs 15.3.1:sdg-15-3-1}}\label{\detokenize{SDGs 15.3.1::doc}}
\sphinxAtStartPar
The Sustainable Development Goal (SDG) 15.3 aims to \sphinxstylestrong{combat desertification} and \sphinxstylestrong{restore degraded land and soil}, striving to achieve a \sphinxstylestrong{land degradation\textendash{}neutral world by 2030}.

\sphinxAtStartPar
Indicator \sphinxstylestrong{15.3.1} measures the \sphinxstylestrong{proportion of degraded land} over the total land area, using three key sub\sphinxhyphen{}indicators:

\begin{sphinxcontents}
\begin{itemize}
\item {} 
\sphinxAtStartPar
\phantomsection\label{\detokenize{SDGs 15.3.1:id7}}{\hyperref[\detokenize{SDGs 15.3.1:land-cover}]{\sphinxcrossref{Land Cover}}}

\item {} 
\sphinxAtStartPar
\phantomsection\label{\detokenize{SDGs 15.3.1:id8}}{\hyperref[\detokenize{SDGs 15.3.1:land-cover-change}]{\sphinxcrossref{Land Cover Change}}}

\item {} 
\sphinxAtStartPar
\phantomsection\label{\detokenize{SDGs 15.3.1:id9}}{\hyperref[\detokenize{SDGs 15.3.1:land-productivity}]{\sphinxcrossref{Land Productivity}}}

\item {} 
\sphinxAtStartPar
\phantomsection\label{\detokenize{SDGs 15.3.1:id10}}{\hyperref[\detokenize{SDGs 15.3.1:carbon-stock}]{\sphinxcrossref{Carbon Stock}}}

\end{itemize}
\end{sphinxcontents}

\sphinxAtStartPar
—


\section{Land Cover}
\label{\detokenize{SDGs 15.3.1:land-cover}}\label{\detokenize{SDGs 15.3.1:id1}}
\begin{figure}[H]
\centering
\capstart

\noindent\sphinxincludegraphics[width=700\sphinxpxdimen]{{land_cover_niger}.png}
\caption{Land cover map for Niger 2009}\label{\detokenize{SDGs 15.3.1:id5}}\end{figure}

\sphinxAtStartPar
The land cover sub\sphinxhyphen{}indicator measures changes in the Earth’s surface cover over time, detecting conversions between natural and human\sphinxhyphen{}modified land types.
\sphinxstylestrong{Typical data sources:}
\sphinxhyphen{} ESA CCI Land Cover

\sphinxAtStartPar
{[}  Back to top{]}(\#sdg\sphinxhyphen{}15\sphinxhyphen{}3\sphinxhyphen{}1\textendash{}proportion\sphinxhyphen{}of\sphinxhyphen{}degraded\sphinxhyphen{}land)

\sphinxAtStartPar
—


\section{Land Cover Change}
\label{\detokenize{SDGs 15.3.1:land-cover-change}}\label{\detokenize{SDGs 15.3.1:id2}}
\begin{figure}[H]
\centering
\capstart

\noindent\sphinxincludegraphics[width=700\sphinxpxdimen]{{landcover_chnage_niger_baseline}.png}
\caption{Land cover map for Niger 2009}\label{\detokenize{SDGs 15.3.1:id6}}\end{figure}

\sphinxAtStartPar
This sub\sphinxhyphen{}indicator assesses \sphinxstylestrong{changes in land cover} between two reference periods (e.g., 2000\textendash{}2015 or 2015\textendash{}2020).
It helps detect transitions between land\sphinxhyphen{}cover classes (forest, cropland, grassland, built\sphinxhyphen{}up areas, etc.) and quantify \sphinxstylestrong{losses or gains of natural land}.

\sphinxAtStartPar
\sphinxstylestrong{Typical data sources:}
\sphinxhyphen{} ESA CCI Land Cover

\sphinxAtStartPar
{[}  Back to top{]}(\#sdg\sphinxhyphen{}15\sphinxhyphen{}3\sphinxhyphen{}1\textendash{}proportion\sphinxhyphen{}of\sphinxhyphen{}degraded\sphinxhyphen{}land)

\sphinxAtStartPar
—


\section{Land Productivity}
\label{\detokenize{SDGs 15.3.1:land-productivity}}\label{\detokenize{SDGs 15.3.1:id3}}
\noindent{\hspace*{\fill}\sphinxincludegraphics[width=600\sphinxpxdimen]{{landcover_chnage_niger_baseline}.png}\hspace*{\fill}}

\sphinxAtStartPar
This sub\sphinxhyphen{}indicator analyzes \sphinxstylestrong{vegetation dynamics} (e.g., NDVI or EVI) to estimate \sphinxstylestrong{productivity trends}.
A persistent decline in vegetation productivity can indicate \sphinxstylestrong{ecological degradation} or \sphinxstylestrong{human\sphinxhyphen{}induced land stress}.

\sphinxAtStartPar
\sphinxstylestrong{Common methods:}
\sphinxhyphen{} NDVI trend analysis (Mann\sphinxhyphen{}Kendall, Theil\sphinxhyphen{}Sen)
\sphinxhyphen{} Deviation from potential productivity
\sphinxhyphen{} Land productivity state classification

\sphinxAtStartPar
\sphinxstylestrong{Data sources:}
\sphinxhyphen{} MODIS NDVI/EVI
\sphinxhyphen{} Landsat NDVI
\sphinxhyphen{} Google Earth Engine analyses

\sphinxAtStartPar
{[}  Back to top{]}(\#sdg\sphinxhyphen{}15\sphinxhyphen{}3\sphinxhyphen{}1\textendash{}proportion\sphinxhyphen{}of\sphinxhyphen{}degraded\sphinxhyphen{}land)

\sphinxAtStartPar
—


\section{Carbon Stock}
\label{\detokenize{SDGs 15.3.1:carbon-stock}}\label{\detokenize{SDGs 15.3.1:id4}}
\noindent{\hspace*{\fill}\sphinxincludegraphics[width=600\sphinxpxdimen]{{landcover_chnage_niger_baseline}.png}\hspace*{\fill}}

\sphinxAtStartPar
The \sphinxstylestrong{carbon stock} sub\sphinxhyphen{}indicator evaluates changes in \sphinxstylestrong{soil and biomass organic carbon}.
Decreases in carbon storage often indicate \sphinxstylestrong{loss of organic matter}, \sphinxstylestrong{deforestation}, or \sphinxstylestrong{soil degradation}.

\sphinxAtStartPar
\sphinxstylestrong{Components considered:}
\sphinxhyphen{} Soil Organic Carbon (SOC)
\sphinxhyphen{} Above\sphinxhyphen{} and below\sphinxhyphen{}ground biomass carbon
\sphinxhyphen{} Total change between two time periods

\sphinxAtStartPar
\sphinxstylestrong{Data sources:}
\sphinxhyphen{} FAO GSOCmap
\sphinxhyphen{} GlobBiomass
\sphinxhyphen{} ESA CCI Biomass

\sphinxAtStartPar
{[}  Back to top{]}(\#sdg\sphinxhyphen{}15\sphinxhyphen{}3\sphinxhyphen{}1\textendash{}proportion\sphinxhyphen{}of\sphinxhyphen{}degraded\sphinxhyphen{}land)



\renewcommand{\indexname}{Index}
\printindex
\end{document}